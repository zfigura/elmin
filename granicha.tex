\documentclass{article}
\usepackage[english]{babel}
\usepackage{tipa}
\usepackage[svgnames]{xcolor}
\usepackage[T1]{fontenc}
\usepackage{verse}

%\usepackage{fontspec}
%\setmainfont{Libertinus Serif}

% partial credit to Steven B. Segletes on stack exchange for this trick
\let\oldthefootnote\thefootnote
\newcommand\oocfootnote[2][DarkGreen]{\renewcommand\thefootnote{\color{#1}\oldthefootnote}%
  \footnote{\color{#1}#2}%
  \renewcommand{\thefootnote}{\oldthefootnote}}

\hyphenation{Yang-harad pred-ic-table}

\begin{document}

\color{DarkGreen}

\section*{Preface}

The following is the original text of a Maotic treatise published by Chásur, namely, Ûchiker's \emph{Taol 16}. The treatise is simply a collection of rituals of various forms, without analysis and with only minimal explanatory commentary.

Out of universe, the rituals are not of my own invention, but are instead direct translations of the text of the twelve songs of Disparition's album \emph{Granicha} (excluding, of course, the entirely instrumental ``Poem for a Crack in the Wall'' and ``Poem for the Unspoken''). These songs are, as stated by Disparition, rituals, largely concerned with water and especially rivers, and this seemed fitting for the Taol, a people whose name literally derives from a word whose in-universe meaning is ``river'', whose land is inundated with rivers, and for whom rivers are a central aspect of life. The songs of \emph{Granicha} are also, as the title might suggest, concerned with borders (Slavic \textit{*granica} ``border''), although this is distinctly less important to the Taol. The actual text of the songs of \emph{Granicha}, and the clarity or obscurity with which these themes are portrayed, varies wildly, but I feel that they are not out of place as ritual texts for a constructed culture.

Perhaps more importantly, though, I am quite fond of the album musically, and I appreciate it lyrically as well. I find its lyrics a reasonable source of translation, and hence they are now Maotic rituals.

Some of the translated text warrants some special notice, where I have replaced English names with Maotic analogues. In the songs ``A Fire in the Distant Hills'' (the sixth ritual listed) and ``Song for the Other Side'' (the eighth) the names of real rivers are given: Danube, Rio Grande, Dniester, Severn. I have replaced these respectively with Elmin rivers: Molcrá, Úrol, Septé, Dzao.

The other notable difficulty in translation is with the third song, ``Doggerland''. As the name suggests, the song concerns the now sunken land which once connected the British Isles to the mainland of Europe. Its text includes the line ``You were there, where R\={e}nos met Tameses'' — that is, where the Rhine met the Thames, a place which only existed in Doggerland. Needless to say, there is no similar place in Elmincár, and I do not wish to introduce one simply to satisfy a translation. Reference to the fall of Dalia could be made, but this took place only a few centuries before the present, and would not match the tone of Doggerland, which was submerged in prehistoric times.

What \emph{is} prehistoric in Elmincár, however, is the creation of the world, or more crucially the modern form thereof. Legend states that the Édris Elmin did not always exist, but were raised by the gods myriad years ago. The legends are vague as to whether this happened during the creation of the world or long after it, but some do believe the latter, and that once the river Maoth flowed west — backwards! — to meet the Peréx.

\color{black}

\section{\oocfootnote{From ``Hexe''.}}

[A river-ritual from the town of Agrókoth on the Cabor, XXX west of Chágáb.]

\begin{verse}
Ód harmó\oocfootnote{Lit. ``beginning''. The usual word for ``source (of a river)'' is \emph{khalun} ``head'', but the next line also uses \emph{khalun}, and in order to avoid repetition where not warranted I've translated this using a different word.} mon rín \\
Rot khalun\oocfootnote{Lit. ``head''. Maotic uses the same idiom as English here.} intéin \\
Met omin cac ash met ódetor ash \\

Dekón met pichin rot rot\oocfootnote{\emph{rot} ``in, at'' is also a general relative pronoun. Hence \emph{rot rot} frequently matches English ``where'', being more literally ``in [the place] that''.} honaon cet ciluák \\!

Unict met napó ot ládró amat\oocfootnote{Lit. ``tubes of blood''.} \\
Umetó tâ ot doág \\
Unict met napó ot ládró amat \\
Umetó tâ ot doág \\!

Lidaol harkol in cilárin cet ciluák ruelinis \\
Líkok harkol in kuvueín cet tûró nakh \\
Umín míné\oocfootnote{Lit. ``amount of a hand'', or even more literally ``weight of a hand'', but using \emph{umín} in this manner is a common Maotic idiom.} ashí \\
Upín é gint \\
Umín inao nain met istônt dékevis \\!

Kancok met vaorok ot ládró amat \\
Akuc rot iliol \\
Akuc rot iliol \\
Akuc rot iliol \\!

Hetor hárn mol in en \\!

Okás okás\oocfootnote{Repeated for emphasis.} upín taló \\
Kuvuín dôró \\
Îrin lúnil ód cocí \\
Pichin vaorok ládál aompanó atlas met vel lîrta\oocfootnote{Lit. ``without end''. The original word here in ``Hexe'' is ``irrational'', but extant mathematics in Elmincár—while not unsophisticated—lacks theoretical exploration of that particular concept, and so I have substituted another.} \\!

Míné in taphin hor ala koth \\
Umín ogot les \\
Rot monuen en nakh líkok\oocfootnote{The original text in ``Hexe'' is ``weight of the waters / crushing the words'', and in isolation this would probably be translated \emph{monuen umín in en líkok}—note the omission of \emph{rot}, as well as the use of \emph{umín in} to imply a more literal ``weight (of)'' rather than the idiomatic uses seen above. However, with \emph{umín} used above—even without \emph{in}—I chose a different formulation to avoid repetition.} \\
Umetó intéin in ód címin dan \\!

\end{verse}

\section{\oocfootnote{From ``Fault''.}}

[An Amain ritual reported to me indirectly from a source in Ínalca on the Maotel. The text is rather obscure and, given the unverifiability of the source, may be incorrect.]

% FIXME ZF: Amain has some dialectal features. What are they? Adjust this accordingly. Though honestly it's probably not much, Amain isn't that out there. Also spelling isn't going to change at all.

\begin{verse}
Phé in maon ál îk \\
Phé in maon ál îk \\
Gaon in maon é intéin \\
Gât in lespin ho udrí lidaol\footnote{[Given as one line, but obviously this does not make sense. Is this an error for ``Ho udrí lidaol gât in lespin''?]} \\
Rín in nuekin [met]\footnote{[Not present in the original source, but this must have been omitted.]} in picacén \\
Vaorok met nain [met?] evos in ní\footnote{[This line is obscure, and the following only confuses the issue further. Is the former a corruption of the latter? Neither makes sense as given.]} \\
Vaorok in nain [met?] evos etamor met in ní \\
Kancok uronó ód kancok uronó horis in ó lor \\
Uronó ód kancok uronó horis in míné \\
Aíl pár min\oocfootnote{Lit ``heart''; the connotation is the same as English.} okoci ot míné rot lómin \\
Aíl pár okoci min okoci [ot]\footnote{[Not present, but must have been omitted.]} míné rot lómin \\
% not done

\end{verse}

\section{\oocfootnote{From ``Doggerland''.}}

% todo: provenance

\begin{verse}
\end{verse}


\section{\oocfootnote{From ``Shadow in the Door''.}}

% todo: provenance

\begin{verse}
\end{verse}


\section{\oocfootnote{From ``The People Who Carry Their Forest Around With Them''.}}

% todo: provenance

\begin{verse}
\end{verse}


\section{\oocfootnote{From ``A Fire in the Distant Hills''.}}

[A common river-ritual found in a relatively wide range near the confluence of the Cabor. The below is an aggregate of two sources: one from Háré on the Maoth, about twenty miles west of Aspen, and the other from Inner Aspen.]
% todo: provenance

\begin{verse}
\end{verse}


\section{\oocfootnote{From ``Analog''.}}

% todo: provenance

\begin{verse}
\end{verse}


\section{\oocfootnote{From ``Song for the Other Side''.}}

% todo: provenance

\begin{verse}
\end{verse}


\section{\oocfootnote{From ``Root Under Stone''.}}

% todo: provenance

\begin{verse}
\end{verse}


\section{\oocfootnote{From ``Book of Arrows''.}}

% todo: provenance

\begin{verse}
\end{verse}


\section{\oocfootnote{From ``Land''.}}

% todo: provenance

\begin{verse}
\end{verse}


\section{\oocfootnote{From ``Iron Circle''.}}

% todo: provenance

\begin{verse}
% not done

% This should be archaïzing, in general. So should the other bit from Doggerland that Marnie Breckenridge sings on. But I know nothing about Maotic grammar diachronically :-(

Tûró ál cet \\
Tûró é cocí \\
Miphon kónót é miphirek \\
Lídirí\oocfootnote{Archaic -í plural of \emph{lídis} ``dancer''.} mukhol ep cadrág\oocfootnote{\emph{Mukhol} can generally be used as the head of a phrase without \emph{in}.} \\
Gint lophant ho udrí \\
\hspace{1em} met edrel in réd \\
\hspace{1em} mób vaí lu \\
Cilín tillin hetor\oocfootnote{Usually translated as ``as'', but unlike English ``as'', \emph{hetor} doesn't necessarily have a metaphorical force.} dés rot mánok cívél \\
\hspace{1em} címin kevekan \\
\hspace{1em} sis ninkua

\end{verse}

\end{document}
