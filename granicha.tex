\documentclass{article}
\usepackage[english]{babel}
\usepackage{tipa}
\usepackage[svgnames]{xcolor}
\usepackage[T1]{fontenc}
\usepackage{verse}

%\usepackage{fontspec}
%\setmainfont{Libertinus Serif}

% partial credit to Steven B. Segletes on stack exchange for this trick
\let\oldthefootnote\thefootnote
\newcommand\oocfootnote[2][DarkGreen]{\renewcommand\thefootnote{\color{#1}\oldthefootnote}%
  \footnote{\color{#1}#2}%
  \renewcommand{\thefootnote}{\oldthefootnote}}

\hyphenation{Yang-harad pred-ic-table}

\begin{document}

\color{DarkGreen}

\section*{Preface}

The following is the original text of a Maotic treatise published by Chásur, namely, Ûchiker's \emph{Taol 16}. The treatise is simply a collection of rituals of various forms, without analysis and with only minimal explanatory commentary.

Out of universe, the rituals are not of my own invention, but are instead direct translations of the text of the twelve songs of Disparition's album \emph{Granicha} (excluding, of course, the entirely instrumental ``Poem for a Crack in the Wall'' and ``Poem for the Unspoken''). These songs are, as stated by Disparition, rituals, largely concerned with water and especially rivers, and this seemed fitting for the Taol, a people whose name literally derives from a word whose in-universe meaning is ``river'', whose land is inundated with rivers, and for whom rivers are a central aspect of life. The songs of \emph{Granicha} are also, as the title might suggest, concerned with borders (Slavic \textit{*granica} ``border''), although this is distinctly less important to the Taol. The actual text of the songs of \emph{Granicha}, and the clarity or obscurity with which these themes are portrayed, varies wildly, but I feel that they are not out of place as ritual texts for a constructed culture.

Perhaps more importantly, though, I am quite fond of the album musically, and I appreciate it lyrically as well. I find its lyrics a reasonable source of translation, and hence they are now Maotic rituals.

Some of the translated text warrants some special notice, where I have replaced English names with Maotic analogues. In the songs ``A Fire in the Distant Hills'' (the sixth ritual listed) and ``Song for the Other Side'' (the eighth) the names of real rivers are given: Danube, Rio Grande, Dniester, Severn. I have replaced these respectively with Elmin rivers: Molcrá, Úrol, Septé, Dzao.

The other notable difficulty in translation is with the third song, ``Doggerland''. As the name suggests, the song concerns the now sunken land which once connected the British Isles to the mainland of Europe. Its text includes the line ``You were there, where R\={e}nos met Tameses'' — that is, where the Rhine met the Thames, a place which only existed in Doggerland. Needless to say, there is no similar place in Elmincár, and I do not wish to introduce one simply to satisfy a translation. Reference to the fall of Dalia could be made, but this took place only a few centuries before the present, and would not match the tone of Doggerland, which was submerged in prehistoric times.

What \emph{is} prehistoric in Elmincár, however, is the creation of the world, or more crucially the modern form thereof. Legend states that the Édris Elmin did not always exist, but were raised by the gods myriad years ago. The legends are vague as to whether this happened during the creation of the world or long after it, but some do believe the latter, and that once the river Maoth flowed west — backwards! — to meet the Peréx.

\color{black}

\section{\oocfootnote{From ``Hexe''.}}

[A river ritual from the town of Agrókoth on the Cabor, XXX west of Chágao.

Although the content seems mostly straightforward (barring one or two more obscure lines), mostly being river veneration, the source of this ritual informs me that it is primarily spoken as an invocation for the dissolution of borders and territories.]

\begin{verse}
Hó khalun an tillin \\
Ol petó êó \\
Né omin ao ash né hóagi ash \\!

Ámoín né acil ol ol\oocfootnote{\emph{ol} ``in, at'' is also a general relative pronoun. Hence \emph{ol ol} frequently matches English ``where'', being more literally ``in [the place] that''.} buen el ives \\!

Ecó né ducó 'l opláló\oocfootnote{Lit. ``tube'', with the connection of connotation, and is the usual word for a tube in the body (artery, vein, trachea) or even the hollow stem of a plant such as a dandelion, but generally not e.g. a reed. \\ The original word is ``artery'', which is more specific. In a more technical context the phrase \emph{opláló 'klon} ``artery'', lit. ``tube of blood'' might be used (contrasting with \emph{ádól} ``vein''). The poetic intent of ``Hexe'', in my analysis, is that ``arteries'' is a metaphor for rivers, and that the choice of ``arteries'' over ``veins'' (which is more poetic and more often metaphorical in English) is probably to emphasize the \emph{bearing} nature—arteries of course being the bearers of oxygen. Since the Maotic word \emph{opláló} alone has as much metaphorical force, I've used it here. } \\
Las hokte ol aola \\
Ecó né ducó 'l opláló \\
Las hokte ol aola \\!

Ílao gotkol in âmin el ives ii \\
Olin gotkol in haín el déló pél \\
Lal haon\oocfootnote{Lit. ``amount of a hand'', or even more literally ``weight of a hand'', but using \emph{lal} in this manner is a common Maotic idiom.} ashí \\
Daor ro mithe \\
Lal ail rophin né amó déksevis \\!

Ataopo né dóg' ol opláló \\
Pao 'l ilín\oocfootnote{Lit. ``in the middle''.} \\
Pao 'l ilín \\
Pao 'l ilín \\!

Agi égachol ren in édal \\!

Okás okás\oocfootnote{Repeated for emphasis.} upín taló \\
Háin déló \\
Arue lúnil hó cocí \\
Acil dógo in oplaon ol álí atlas né fel theto\oocfootnote{Lit. ``without end''. The original word here in ``Hexe'' is ``irrational'', but extant mathematics in Elmincár—while not unsophisticated—lacks theoretical exploration of that particular concept, and so I have substituted another. Unfortunately attempting to preserve the same metaphorical force is difficult, since this is easily the most obscure line in the song.} \\!

Haon in otcotor nís ilo ko \\
Lal huedin tés \\
Ol moin édal las olin\oocfootnote{The original text in ``Hexe'' is ``weight of the waters / crushing the words'', and in isolation this would probably be translated \emph{moin lal in édal olin}—note the omission of \emph{ol}, as well as the use of \emph{lal in} to imply a more literal ``weight (of)'' rather than the idiomatic uses seen above. However, with \emph{lal} used above—even without \emph{in}—I chose a different formulation to avoid repetition.} \\
Las êó in haer hó áglo \\!

\end{verse}

\section{\oocfootnote{From ``Fault''.}}

[An Amain ritual reported to me indirectly from a source in Ínalca on the Maotel. The text is rather obscure and, given the unverifiability of the source, may be incorrect.]

% FIXME ZF: Amain has some dialectal features. What are they? Adjust this accordingly. Though honestly it's probably not much, Amain isn't that out there. Also spelling isn't going to change at all.

\begin{verse}
Gus in hiin ol reve \\
Gus in hiin ol reve \\
Álos in hiin ró êó \\
Aoguebin in âcsin ko tus gucin\footnote{[Given as one line, but obviously this does not make sense. Is this an error for ``Ko tus gucin aoguebin in âcsin''?]} \\
Tillin in rûn [né]\footnote{[Not present in the original source, but this must have been omitted.]} in acicén \\
Dógo né rophin oknotlos in fos\footnote{[This line is obscure, and the following only confuses the issue further. Is the former a corruption of the latter? Neither makes sense as given.]} \\
Dógó in rophin oknotlos humor né in fos \\

Atopo riló ho atopo riló níris in ó phus \\
Riló ho atopo riló níris in lal haon \\
Amor r' ankas\oocfootnote{Lit ``heart''; the connotation is the same as English.} kén thes haon ol lakin \\
Amor raci ankas kén [thes]\footnote{[Presumably omitted, and should be supplied from the previous line.]} haon kén ol lakin \\

% not done

\end{verse}

\section{\oocfootnote{From ``Doggerland''.}}

% todo: provenance

\begin{verse}
\end{verse}


\section{\oocfootnote{From ``Shadow in the Door''.}}

% todo: provenance

\begin{verse}
\end{verse}


\section{\oocfootnote{From ``The People Who Carry Their Forest Around With Them''.}}

% todo: provenance

\begin{verse}
\end{verse}


\section{\oocfootnote{From ``A Fire in the Distant Hills''.}}

[A common river-ritual found in a relatively wide range near the confluence of the Cabor. The below is an aggregate of two sources: one from Háré on the Maoth, about twenty miles west of Aspen, and the other from Inner Aspen.]
% todo: provenance

\begin{verse}
\end{verse}


\section{\oocfootnote{From ``Analog''.}}

% todo: provenance

\begin{verse}
\end{verse}


\section{\oocfootnote{From ``Song for the Other Side''.}}

% todo: provenance

\begin{verse}
\end{verse}


\section{\oocfootnote{From ``Root Under Stone''.}}

% todo: provenance

\begin{verse}
\end{verse}


\section{\oocfootnote{From ``Book of Arrows''.}}

% todo: provenance

\begin{verse}
\end{verse}


\section{\oocfootnote{From ``Land''.}}

% todo: provenance

\begin{verse}
\end{verse}


\section{\oocfootnote{From ``Iron Circle''.}}

% todo: provenance

\begin{verse}
% not done

% This should be archaïzing, in general. So should the other bit from Doggerland that Marnie Breckenridge sings on. But I know nothing about Maotic grammar diachronically :-(

Déló 'l édal \\
Déló ró cocí \\
Aomais fucol ró buce \\
Luedirí\oocfootnote{Archaic -í plural of \emph{luedis} ``dancer'', which is itself a fossilized and slightly poetic agent noun.} nís\oocfootnote{A somewhat archaic choice of preposition; modern Maotic would normally use \emph{ol} here.} édas cokhó módo \\
Mithe toptha ko tus \\
\hspace{1em} né ocus in cuu \\
\hspace{1em} mó fâs lu \\
Ruelo tillin agi\oocfootnote{Usually translated as ``as'', but unlike English ``as'', \emph{agi} doesn't necessarily have a metaphorical force.} élén ol éduetkin\footnote{[Archaic verbal form of ``édue''.]} \\
\hspace{1em} haer kevekan \\
\hspace{1em} sis ninkua \\!

\end{verse}

\end{document}
