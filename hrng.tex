\documentclass[a4paper]{article}
\usepackage[english]{babel}
\usepackage{tipa}
%\usepackage[margin=1in]{geometry}
\usepackage[onehalfspacing]{setspace}
\usepackage[skip=12.0pt plus 2.0pt]{parskip}

\begin{document}

The following excerpts are taken from treatises published by Chásur\footnote{Chásur (Classical Maotic \textipa{/k\super{j}\super{h}a:suK/}, Modern \textipa{/C\super{h}as5/} or Dálx \textipa{/h\textsubumlaut{E}sO/}) is the location of an institute focused primarily on anthropological study and publication. The institute was founded in Fírecár in the city of Chásur (a short distance southwest of the larger city of XXX) by a small group of young Talócár aristocrats, mostly from Akol Bílé and surrounding areas, who were enthralled by Fírecár language and culture. \\ For multiple decades the institute produced treatises centered around Fírecár; however, anthropological curiosity left unchecked will soon consume anything it can see, and combined with the attraction of Fírecár natives similarly interested in Talócár and Agrócár the institute soon began publishing many treatises concerned with study of all parts of Elmincár. \\ Treatises are published almost exclusively in a (slightly archaïzing) version of Modern Maotic; most early publications are also translated into (Modern) Baerloni, the native language of XXX.}. The excerpts are cited by title, treatise number, and author. The somewhat unfortunate practice for academic publications established by Chásur is to give the treatise a single word title describing the subject in the broadest sense, followed by a number distinguishing which identically titled publication by the same author is being cited.

\end{document}
