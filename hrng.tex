\documentclass{article}
\usepackage[english]{babel}
\usepackage{tipa}
\usepackage[svgnames]{xcolor}
%\usepackage[margin=1in]{geometry}
%\usepackage[onehalfspacing]{setspace}
%\usepackage[skip=12.0pt plus 2.0pt]{parskip}

\begin{document}

\color{DarkGreen}

\section*{Preface}

The following excerpts are taken from treatises published by Chásur\footnote{\color{DarkGreen} Chásur (Classical Maotic \textipa{/k\super{j}\super{h}a:suK/}, Modern \textipa{/C\super{h}as5/} or Dálx \textipa{/h\textsubumlaut{E}:sO/}) is the location of an institute focused primarily on anthropological study and publication. The institute was founded in Fírecár in the city of Chásur (a short distance southwest of the larger city of Andu) by a small group of young Talócár aristocrats, mostly from Akol Bílé and surrounding areas, who were enthralled by Fírecár language and culture. \par For multiple decades the institute produced treatises centered around Fírecár; however, anthropological curiosity left unchecked will soon consume anything it can see, and combined with the attraction of Fírecár natives similarly interested in Talócár and Agrócár the institute soon began publishing many treatises concerned with study of all parts of Elmincár. \par Treatises are published almost exclusively in a (slightly archaïzing) version of Modern Maotic; most early publications are also translated into (Modern) Baerloni, the native language of Andu.}. The excerpts are cited by title, treatise number, and author. The somewhat unfortunate practice for academic publications established by Chásur is to give the treatise a single word title describing the subject in the broadest sense (which, for anthropological texts such as those quoted here, is often a place name), followed by a number distinguishing which identically titled publication by the same author is being cited. To make matters worse, the name of the publication is often omitted when it can be inferred from the previous text.

The treatises quoted in this document are all written in Modern Maotic. Translation convention is applied except for proper names; while many Maotic names have a transparent etymology (e.g. \textit{Fírecár} = ``Green Land'', \textit{Talócár} = ``River Land'', \textit{Actó Atlas} = ``Great Rift''), it hardly seems fitting to show off a conworld and its associated conlangs but neglect a simple opportunity to provide extra examples of these.

\section*{History of Yangharad}

(From Ólófar, \textit{Yangharad 5})

\color{black}
\quotation{

Although the historical settlement of eastern Agrócár has been better and more completely summarized elsewhere (Gant 9, Amíol 1, Ólófar 10), it behoves us to briefly discuss the surrounding areas and their historical cultural composition.

The swath of land between Agróme Toléith and the southeastern spur of the Édris Elmin---and extending far into terraneous and subterraneous mountain settlements---is populated by many small varities of what must clearly have been a single original Mittelo-Tavarian culture, however spatially or chronologically limited this culture was. With rare exception the languages can be systematically traced back at a Classical period to Mittelic or to High or Low Tavarian (Ruecán: Agrócár 27). Cultural similarities other than language are harder to measure and easier to borrow (and hence often misleading, as in the case of Yangharad itself—cf. Taocán 8) but generally support the theory of a unified Mittelo-Tavarian culture distinct from surrounding (inner Étéic, Agróme Toléith, Taol) cultures, measured in terms of art, technology, and magical practices (for a complete catalogue of comparative analyses, see Ólófar: Agrócár 13).

This is not to say that eastern Agrócár was unified politically, or even that it was unified culturally or linguistically on a surface level; indeed all manner of written records show that it was anything but. During the recorded history of Dalia there are recorded several dozen individual conflicts of varying severity between individual Agrócár states, not counting external wars with Dalia or Talócár (Gant 4, 8; Suémin 2, 3, 8). These wars were usually fought for one of two things: either control of the trade road that ran from Aspen and Maotel west, through Dalia and Muktai, through Agróme Toléith and on west to Ash Cásper, or an attempt to unify eastern Agrócár ``from the forest to the mountains.'' Attempts at the latter were rarely successful, and rarely for long.

After the fall of Dalia the landscape of eastern Agrócár changed. For multiple centuries there was no easy access to Talócár, and no Dalian prosperity to reap the benefits of. At the same time records show a similar decrease in trade and contact with western Agrócár, for unclear reasons (Zûtsimazôn 4). By all accounts one of the two principal impetuses for warfare in eastern Agrócár had disappeared, and indeed after the fall of Dalia the formation of Tavarian states and conflict therebetween was no longer ubiquitous (Suémin 8).

It is a popular although apocryphal story that the entire mass of Dalians, having witnessed but escaped the disapperance of their kingdom, were led away by their king and departed east over the Ash Alóné, ``never to be seen again in Elmincár.'' The educated skeptic, on the other hand, will invariably assert that the population of Dalia was simply wiped off the face of Elmincár by the gods along with the kingdom itself, or that they, the halls of Dalia, and indeed the entire cirque, were buried under mountains thrown up by the gods. After studying the region one can believe neither account much if at all; indeed a far simpler explanation is that the great majority of Dalian aristocrats migrated west and took refuge in what were once Tavarian and Mittelic settlements. The popular account that eastern Agrócár was conquered by the Maotic empire at this time is not only incredulous (given the difficulty in crossing the Actó Atlas) but quite simply unnecessary; the Dalians were Maotic speakers, and shared the same literary and artistic traditions as Maotel and indeed most if not all of southern Talócár.\footnote{For a more comprehensive argument in favor of a Dalian ``colonization'' of eastern Agrócár, see Ólófar: Agrócár 9 and 10.}

In light of this theory it should come as no surprise that those settlements closest to Dalia have a most prominently Maotic composition, especially among the educated classes, to the point that many Tavarian and Mittelic dialects are rarely spoken, unstandardized, and often moribund (Ruecán: High Tavarian 7, Agrócár 27). In Muktai and Avalos, Tavarian dialects are all but extinct (ibid.); by contrast, Sevros boasts a strong bilinguality between Maotic and Inner Mittelic (Ruecán 2), and the mountain cities of Qhuraal and Nakhêqe nearly three hundred kilometres southeast of Dalia have a majority of non-Maotic speakers (Ruecán: Qhuraal 1 and 2; Myrve 1).\footnote{To some degree this would be the expected distribution even if these Maotic speakers really had been instated due to conquest originating in Talócár proper. However, I argue that one would expect less of a consistent gradient in the concentration of Maotic speakers; it does not make much sense that Maotel would attempt to unify eastern Agrócár but fail to exert an equal level of control on all of its cities. The situation of Mentos (v. supr.) is also instructive.}

Of special mention is Mentos, which was no more geographically close to Dalia than Sevros but has a disproportionately high concentration of Maotic speakers; this is likely due to its situation on Namó Ablés and associated access to ambient magical energy. Dalia was known to be a center of magical study and artifice as well as one of material wealth (Suémin 13, 23); Dalian sorcerors and enchanters would likely have sought out the next best location of magical practice after its fall.


Given all of this background, it becomes a rather anticlimactic revelation that the city of Yangharad was uninvolved in these disputes, and has remained
independent for over a millennium, as far back as the recorded history of Dalia if not earlier. It is a similarly anticlimactic revelation that while a population of Maotic speakers does exist in Yangharad, it is a minority relative to that of native Yangharad speakers.\footnote{In fact, the proportion of Maotic speakers in Yangharad, XXX percent (Ólófar 2), is actually less than that of Qhuraal, XXX percent (Ruecán 1); given the relative distances of the two to Dalia we would expect the opposite.} Yet it is crucial to understand the entire political landscape, lest obvious questions such as "why were the Yangharad \emph{not} supplanted by Taol or Mittelo-Tavarians?" go unanswered.


Besides being in an obviously well-defended position, Yangharad was never
wealthy enough to be a worthwhile target for conquest or tribute. The state was not exactly rich in natural resources. It sustained itself through terraced farming, in almost all cases requiring magical assistance to obtain even minimally arable soil (Ólófar 1); while the entire span of the Actó Atlas is rich enough in ambient magic to provide a source for such enchantment, Yangharad was hardly the magical center of the world either (indeed, that quite literally lies over four hundred kilometres to the northeast). Dalia itself lay on a far richer wellspring of magic, if records of magical artifice from there are any indication;\footnote{Empirical measurements suggest very high levels of ambient magic even today, levels which rival even Namó Ablés, such that only the thorough inaccessibility of the Dalian ruins can explain their barrenness (Ólúcan 39). However Ólúcan (70) argues that the level of ambient magic observable \emph{today} may not have been sustained from before the state's collapse; whether this argument carries weight or not, it is probably an error to assume that the level of ambient magic observable from the surface was equal to (or even less than!) that observable from underground or even from the nadir of the historic cirque, and we must therefore rely only on written evidence.} control of that locale would have been much more desirable.

This is not to say that the fall of Dalia did not affect Yangharad, of course, although it is in ways subtle and subject to some degree of assumption. In the first place, while the mining and metallurgical industries of Dalia were eventually replaced by others spanning much of the Édris Elmin, Yangharad and Uenter and surrounding locales (esp. Loqotane, Awetatane, Wometa, Kantane)\footnote{Yangharad: \d{L}o\.{q}ota\d{n}e, Awe\d{t}ata\d{n}e, Wome\d{t}a, Kanta\d{n}e.} are widely considered the most prosperous of these (Ólófar: Édris Elmin 1). One must conclude that, had Dalia not fallen, Yangharad would not enjoy such prosperity.

To what degree this industry existed before the fall of Dalia is uncertain, of course. As with most of modern Édris Elmin, Yangharad metallurgy is all but defined by Dalian expertise (Ólófar 4). There is however historic evidence of a pre-Classical Yangharad metalworking tradition, which by Classical times was supplanted by imitation of Dalian ware. Even during the Classical period we find frequent copies of Dalian ware that were probably produced outside of Dalia, although we have as yet no way of knowing the provenance of the metals (Isân: Yangharad 1). It is therefore likely, although not certain, that iron and tin mining in Yangharad and surrounding areas dates back at least a millennium, if not longer---but whether it held any significance compared to Dalia is unknown.

In the second place, the clearing of the Kéí pass\footnote{Yangharad: La\.{q}ekaye. The element \textit{la\.{q}e} translates to ``pass''; \textit{Kaye} is a toponym of unclear origin, but probably has the same source as Maotic \textit{Kéí}.} was

}

\end{document}
