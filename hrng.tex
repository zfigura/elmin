\documentclass{article}
\usepackage[english]{babel}
\usepackage{tipa}
\usepackage[svgnames]{xcolor}
\usepackage[T1]{fontenc}

\hyphenation{Yang-harad pred-ic-table}

\begin{document}

\color{DarkGreen}

\section*{Preface}

The following excerpts are taken from treatises published by Chásur.\footnote{\color{DarkGreen} Chásur (Modern Maotic \textipa{/C\super{h}as5/} or Dálx \textipa{/h\textsubumlaut{E}:sO/}) is the location of an institute focused primarily on anthropological study and publication. The institute was founded in Fírecár in the city of Chásur (a short distance southwest of the larger city of Andu) by a small group of young Talócár aristocrats, mostly from Akol Bílé and surrounding areas, who were enthralled by Fírecár language and culture. \par For nearly a decade the institute produced solely treatises centered around Fírecár. However, anthropological curiosity left unchecked will soon consume anything it can see, and combined with the attraction of Fírecár natives similarly interested in Talócár and Agrócár, it was not long before the institute started publishing treatises concerned with all parts of Elmincár. \par Treatises are published almost exclusively in a (slightly archaïzing) version of Modern Maotic. Most early publications are also translated into (Modern) Baerloni, the native language of Andu.} The excerpts are cited by title, treatise number, and author. The somewhat unfortunate practice for academic publications established by Chásur is to give the treatise a single word title describing the subject in the broadest sense (which, for anthropological texts such as those quoted here, is often a place name), followed by a number distinguishing which identically titled publication by the same author is being cited. To make matters worse, the name of the publication is often omitted when it can be inferred from the previous text.

The treatises quoted in this document are all written in Modern Maotic. Translation convention is applied except for proper names; while many Maotic names have a transparent etymology (e.g.\ \textit{Fírecár} = ``Green Land'', \textit{Talócár} = ``River Land'', \textit{Namó Ablés} = ``Sea of Storms''), it hardly seems fitting to show off a conworld and its associated conlangs but neglect a simple opportunity to provide extra examples of these.

\color{DarkGreen}
\section*{Ethnographic History}

(From Ólófar, \textit{Yangharad 5})

\color{black}
\begin{quotation}

Although the historical settlement of eastern Agrócár has been better and more completely summarized elsewhere (Gant 9, Amíol 1, Ólófar 10), it behoves us to briefly discuss the surrounding areas and their historical cultural composition.

The swath of land between Agróme Toléith and the southeastern spur of the Édris Elmin—and extending far into terraneous and subterraneous mountain settlements—is populated by many small varities of what must clearly have been a single original Mittelo-Tavarian culture, however spatially or chronologically limited this culture was. With rare exception the languages can be systematically traced back at a Classical period to Mittelic or to High or Low Tavarian (Ruecán: Agrócár 27). Cultural similarities other than language are harder to measure and easier to borrow (and hence often misleading, as in the case of Yangharad itself—cf. Taocán 8) but generally support the theory of a unified Mittelo-Tavarian culture distinct from surrounding (inner Étéic, Agróme Toléith, Taol) cultures, measured in terms of art, technology, and magical practices (for a complete catalogue of comparative analyses, see Ólófar: Agrócár 13).

This is not to say that eastern Agrócár was unified politically, or even that it was unified culturally or linguistically on a surface level; indeed all manner of written records show that it was anything but. During the recorded history of Dalia there are recorded several dozen individual conflicts of varying severity between individual Agrócár states, not counting external wars with Dalia or Talócár (Gant 4, 8; Suémin 2, 3, 8). These wars were usually fought for one of two things: either control of the trade road that ran from Aspen and Maotel west, through Dalia and Muktai, through Agróme Toléith and on west to Ash Cásper, or an attempt to unify eastern Agrócár ``from the forest to the mountains.'' Attempts at the latter were rarely successful, and rarely for long.

After the fall of Dalia the landscape of eastern Agrócár changed. For multiple centuries there was no easy access to Talócár, and no Dalian prosperity to reap the benefits of. At the same time records show a similar decrease in trade and contact with western Agrócár, for unclear reasons (Zûtsimazôn 4). By all accounts one of the two principal impetuses for warfare in eastern Agrócár had disappeared, and indeed after the fall of Dalia the formation of Tavarian states and conflict therebetween was no longer ubiquitous (Suémin 8).

% FIXME ZF: The below paragraph is awkward. I like it in terms of writing, but it has problems. One is that this is a treatise meant for academics who should already be familiar with theories of Dalia's fall (and I think should largely doubt both popular explanations for what happened to its inhabitants). Another is that it's kind of out of scope to even explain the reasons for Dalia's fall here, although this kind of applies to most of what I've written.

% It's also a problem that Ólófar says ``Dalia can't just have been wiped off the face of the earth'' but doesn't explain why. Well, it's a problem that we don't have a good explanation for this. Maybe we don't need one, if the prevailing theory is that ``Dalia was wiped off the earth and all of the Taol in Agrócár just came from Talócár.'' But I'd like it to be a bit more controversial than that. I just don't know why it *would* be.

It is a popular although apocryphal story that the entire mass of Dalians, having witnessed but escaped the disapperance of their kingdom, were led away by their king and departed east over the Ash Alóné, ``never to be seen again in Elmincár.'' The educated skeptic, on the other hand, will invariably assert that the population of Dalia was simply wiped off the face of Elmincár by the gods along with the kingdom itself, or that they, the halls of Dalia, and indeed the entire cirque, were buried under mountains thrown up by the gods. After studying the region one can believe neither account much if at all; indeed a far simpler explanation is that the great majority of Dalian aristocrats migrated west and took refuge in what were once Tavarian and Mittelic settlements. The popular account that eastern Agrócár was conquered by the Maotic empire at this time is not only incredulous (given the difficulty in crossing the Édris Elmin) but quite simply unnecessary; the Dalians were Maotic speakers, and shared the same literary and artistic traditions as Maotel and indeed most if not all of southern Talócár.\footnote{For a more comprehensive argument in favor of a Dalian ``colonization'' of eastern Agrócár, see Ólófar: Agrócár 9 and 10.}

% It's also possible that we should keep the above paragraph but omit this next one. It's basically evidence in favor of Ólófar's theory. On the other hand it does help to give context, essentially asking the question ``well, are there Taol in Yangharad?'' that the later paragraphs answer.

In light of this theory it should come as no surprise that those settlements closest to Dalia have a most prominently Maotic composition, especially among the educated classes, to the point that many Tavarian and Mittelic dialects are rarely spoken, unstandardized, and often moribund (Ruecán: High Tavarian 7, Agrócár 27). In Muktai and Avalos, Tavarian dialects are all but extinct (ibid.); by contrast, Sevros boasts a strong bilinguality between Maotic and Inner Mittelic (Ruecán 2), and the mountain cities of Qhuraal and Nakhêqe nearly three hundred kilometres southeast of Dalia have a majority of non-Maotic speakers (Ruecán: Qhuraal 1 and 2; Myrve 1).\footnote{To some degree this would be the expected distribution even if these Maotic speakers really had been instated due to conquest originating in Talócár proper. However, I argue that one would expect less of a consistent gradient in the concentration of Maotic speakers; it does not make much sense that Maotel would attempt to unify eastern Agrócár but fail to exert an equal level of control on all of its cities. The situation of Mentos (v. supr.) is also instructive.}

Of special mention is Mentos, which was no more geographically close to Dalia than Sevros but has a disproportionately high concentration of Maotic speakers; this is likely due to its situation on Namó Ablés and associated access to ambient magical energy. Dalia was known to be a center of magical study and artifice as well as one of material wealth (Suémin 13, 23); Dalian sorcerors and enchanters would likely have sought out the next best location of magical practice after its fall.

Given all of this background, it becomes a rather anticlimactic revelation that the city of Yangharad was uninvolved in these disputes, and has remained independent for over a millennium, as far back as the recorded history of Dalia if not earlier. It is a similarly anticlimactic revelation that while a population of Maotic speakers does exist in Yangharad, it is a minority relative to that of native Yangharad speakers.\footnote{In fact, the proportion of Maotic speakers in Yangharad, XXX percent (Ólófar 2), is actually less than that of Qhuraal, XXX percent (Ruecán 1); given the relative distances of the two to Dalia we would expect the opposite.} Yet it is crucial to understand the entire political landscape, lest obvious questions such as ``why were the Yangharad \emph{not} supplanted by Taol or Mittelo-Tavarians?'' go unanswered.

Besides being in an obviously well-defended position, Yangharad was never wealthy enough to be a worthwhile target for conquest or tribute. The state was not exactly rich in natural resources. It sustained itself through terraced farming, in almost all cases requiring magical assistance to obtain even minimally arable soil (Ólófar 1); while the entire span of the Édris Elmin is rich enough in ambient magic to provide a source for such enchantment, Yangharad was hardly the magical center of the world either (indeed, that quite literally lies over four hundred kilometres to the northeast). Dalia itself lay on a far richer wellspring of magic, if records of magical artifice from there are any indication;\footnote{Empirical measurements suggest very high levels of ambient magic even today, levels which rival even Namó Ablés, such that only the thorough inaccessibility of the Dalian ruins can explain their barrenness (Ólúcan 39). However Ólúcan (70) argues that the level of ambient magic observable \emph{today} may not have been sustained from before the state's collapse; whether this argument carries weight or not, it is probably an error to assume that the level of ambient magic observable from the surface was equal to (or even less than!) that observable from underground or even from the nadir of the historic cirque, and we must therefore rely only on written evidence.} control of that locale would have been much more desirable.

This is not to say that the fall of Dalia did not affect Yangharad, of course, although it did so in ways subtle and subject to some degree of assumption. In the first place, while the mining and metallurgical industries of Dalia were eventually replaced by others spanning much of the Édris Elmin, Yangharad and Uenter and surrounding Yangharad-speaking settlements (esp. Loqotane, Awetatane, Wometa, Kantane) are widely considered the most prosperous of these (Ólófar: Édris Elmin 1). One must conclude that, had Dalia not fallen, Yangharad would not enjoy such prosperity.

To what degree this industry existed before the fall of Dalia is uncertain, of course. As with most of modern Édris Elmin, Yangharad metallurgy is all but defined by Dalian expertise (Ólófar 4). There is however historic evidence of a pre-Classical Yangharad metalworking tradition, which by Classical times was supplanted by imitation of Dalian ware. Even during the Classical period we find frequent copies of Dalian ware that were probably produced outside of Dalia, although we have as yet no way of knowing the provenance of the metals (Isân: Yangharad 1). It is therefore likely, although not certain, that iron and tin mining in Yangharad and surrounding areas dates back at least a millennium, if not longer—but whether it held any significance compared to Dalia is unknown.

In the second place, the clearing of the Kéí pass was undoubtedly the most important event of the past two centuries concerning the Yangharad states. It is much more a matter of speculation whether this can be said to be a consequence of the fall of Dalia, but the argument is not hard to see: had there been any other accessible passage between Agrócár and Talócár, the Kéí pass would not have gained the importance that it now holds, and might never have been cleared in the first place. It is worth noting, of course, that the Kéí pass would probably never have been cleared without the assistance of Ácantí transmutation, although it is not clear that this is an argument for or against the importance of Dalia's fall.

Regardless of the reasons, the clearing of the Kéí pass undoubtedly paved the way for the rise of Uenter—and of greater Yangharad—as a center of trade, and ensured the city's prosperity, and hence cultural flourishing. In specific Uenter, which had until that point been of little note\footnote{It is unclear what relation Uenter held to the city of Yangharad in Classical times—whether it was a tributary, or considered an peer of the city under common governance, or truly independent. Unfortunately, during the Classical and Post-Classical periods, documentation from within Yangharad is scarce to nonexistent, and documentation outside of Yangharad rarely mentions other settlements but the city itself.} (Ólófar: Yangharad 2), became the most populous and prominent Yangharad-speaking state, and remains so to this day.

\end{quotation}

\color{DarkGreen}

There is little that I need to add to the above in terms of commentary, although for the sake of clarity it may do to give a brief summary of Dalia itself, irrelevant as it mostly is to Yangharad. I would normally do so by quoting another Chásur treatise, but the history and fall of Dalia is such a highly studied subject that it is hard if at all possible to find a \emph{short} summary of the entire empire's fall.

Dalia is the name given both to a cirque located in the eastern spur of the Édris Elmin, and to a historical kingdom which occupied that cirque, as well as subterranean structures below it and extending into the neighbouring mountains. ``Cirque'' is somewhat of a misnomer here—the actual area spanned dozens of miles in either direction, and was not so steep a valley as one might be thus led to believe—but it at least accurately describes the bowl shape of Dalia, which was a relatively flat lowland area surrounded by mountains on all sides.

Dalia was a kingdom founded on two things: mining and trade. The former included all manner of precious and base metals (as well as, to some degree, jewels), especially iron, silver, and gold. In predictable Tolkienian fashion for a mining kingdom set in a fantasy world, there were significant subterranean cities. However, unlike Moria, the entirety of Dalia was not subterranean; in fact, while the governing bodies of Dalia were located underground, most of the population lived aboveground.

Dalia's prosperity in trade can be clearly credited to its position.  As was briefly mentioned in the above treatise, the cirque lay in the middle of the Édris Elmin, which not only divided Agrócár from Talócár, but also thereby divided the western ocean from the eastern. Furthermore, Dalia essentially \emph{was} the most accessible pass by far between Agrócár and Talócár. Combined with Dalia's access to precious metals (which are otherwise difficult to find in the Édris Elmin), it easily became a formidable trade state.

Several hundred years ago, the entire kingdom of Dalia, and the cirque in which it lay, disappeared, or perhaps it should be said that they were buried. Where the cirque once was mountains now stood; these mountains were as impassable or more as the rest of the Édris Elmin.

The impact of this event on Agrócár is described well enough, albeit in passing, in the above treatise. The \emph{cause} Ólófar does not discuss; this is partly due to its irrelevance, but mostly because it is completely unknown to the inhabitants of Elmincár. Given the nature of the event it is generally believed to be due to divine intervention, but unusually for world-changing events of divine cause, nobody (in-universe) actually knows what happened, and no record has been found of the events that took place.

It is no surprise, then, that ``what happened to Dalia?'' is one of the largest burning questions, if not \emph{the} largest, in the minds of Chásur academics. Given all this information, though, Ólófar's explanation seems odd: one would expect it to either be obvious and well-supported, or completely implausible.

The explanation for this paradox is that while those people can be found who claim Dalian ancestry, they are rare. In particular, despite Ólófar's arguments, they do not include the entire upper class of Agrócár cities. Ólófar argues in Agrócár 9 and 10 that whatever divine event felled Dalia must also have somehow transformed its inhabitants into Agrócárians, in some way or another. There is some faint evidence to support this, but the theory is controversial to say the least, and the evidence can be interpreted in many ways. Perhaps more importantly, there is simply no clearly better explanation for Dalia's disappearance—thus making Ólófar's theories both plausible and not certain.

There is a real explanation of what happened to Dalia and its inhabitants, but it is a long and complex narrative, and there is neither space nor relevance for it here.

\section*{Ethnography}

(From Ólófar, \textit{Yangharad 2})

\color{black}
\begin{quotation}

\textit{The following treatise is written primarily based on the fieldwork of Ruecán, Taocán, and myself.}

\hfill

With the exception of the language itself, there is little that separates the continuum of Yangharad speakers from their Tavarian neighbours. Their art, architecture, clothing, and metallurgy is either indistinguishable from that of their neighbours, or shows no more local variation than can be seen in any other Tavarian city. While the major states of Yangharad and Uenter share many cultural similarities, politically they are not unified, nor do they even have the same structures of government. Even the Yangharad language, while unique in many ways, may easily be compared to Qhuraal or Myrve by the undiscerning eye.

It is no surprise then that early treatises, inasmuch as they concern themselves with Yangharad at all, blithely consider the inhabitants of Yangharad and Uenter to be Tavarians, and the Yangharad language to be a peculiar Tavarian variety. Recent research, however, suggests that there is a coherent Yangharad culture independent from its neighbours. Ruecán's treatment of the language itself (1) and her comparative studies of the Mittelo-Tavarian languages (1, 3) shows without a doubt that Yangharad is not even distantly related to the Mittelo-Tavarian languages, but must have a unique history. Furthermore, in my own studies of Yangharad

\end{quotation}

\color{DarkGreen}

% Should any of the below be replaced with later documents?

The above document, Ólófar's \textit{Yangharad 2}, was made relatively early in a period of concerted and holistic (and, in a sense, revisionist) Tavarian study. Earlier study tended to view the entirety of eastern Agrócár as a politically fragmented, but culturally quite homogeneous, people.

This may be at least partially motivated by relative a lack of interest in Agrócár as a whole. The object of greatest interest in Chásur studies, and in Taol culture generally, has always been Dalia (which, it must be asserted, comes as no surprise; Dalia's existence and its fall had a huge cultural impact not only on Talócár and on the Maotic empire, but on all of Elmincár). Even before Ólófar's time, though, there was certainly study of Tavarian culture and history for its own sake. It is also worth noting that, while there is certainly observable local tradition throughout eastern Agrócár, it \emph{is} a much more culturally homogeneous place than Fírecár or even northern Talócár, so this lack of focus into specific Agrócár ethnicities is not unwarranted.

Ólófar was one of the first to assert the idea of a Yangharad people culturally distinct from the rest of Agrócár (i.e.\ from the Mittelo-Tavarians), and the above treatise is one of the first to explore the concept. Therefore it is worthy of some note that the assertion of cultural similarities between Yangharad and other Mittelo-Tavarian peoples made early in the treatise is based on prior research, and would actually later be challenged in some areas. Later research by Ólófar and others would assert the separation of the Yangharad culture based on musical tradition, as well as the eventual discovery of pre-Dalian artistry in metalware.

\appendix
\pagebreak
\section{Glossary of proper names}

The following is a summary proper names in the above text, including pronunciation in various Elmin languages as well as brief definitions. All names are given pronunciation in standard Modern Maotic as well as Dálx Maotic; where applicable Classical Maotic pronunciation is also specified.

\begin{itemize}
 \item \textit{Agrócár}:

 \item \textit{Agróme Toléith}:

 \item \textit{Akol Bílé}:

 \item \textit{Andu}:

 \item \textit{Ash Alóné}:

 \item \textit{Ash Cásper}:

 \item \textit{Aspen}: A city on the mouth of the Maoth river. One of the largest and most important cities in Talócár. Classical and Modern Maotic \textipa{/aspen/}, Dálx \textipa{/asp\~En/}.

 \item \textit{Avalos}:

 \item \textit{Awetatane}: A Yangharad iron and tin mining town near the city of Uenter. From Yangharad \textit{Awe\d{t}ata\d{n}e}.

 \item \textit{Baerloni}: The most prominent language spoken in southeastern Fírecár, including the major cities of Andu and Akol Bílé. The name is borrowed verbatim from native Baerloni, where it is pronounced \textipa{/(w)a.e\*rloni/} or (dialectal) \textipa{/a.\textrhookschwa{}løni/}. The Maotic pronunciation varies.\footnote{The educated pronunciation is \textipa{/bajr.loni/} or (careful) \textipa{/ba.er.lo.ni/}, but the name is not well suited to Maotic phonotactics, and the \textipa{/r/} is usually dropped. More frequently, though, the syllable-final <r> is interpreted as historic \textipa{/K/}, and the word is pronounced \textipa{/ba.aloni/}, or, with dissimilation assisted by spelling-pronunciation, \textipa{/ba.eloni/}.}

 \item \textit{Chásur}: Location in Talócár of an institute for anthropological study. Modern Maotic \textipa{/C\super{h}as5/}, Dálx \textipa{/h\textsubumlaut{E}:sO/}. From Baerloni \textit{Chasur} \textipa{/ts\super{j}\super{h}a.su\*r/}.\footnote{Although (standard i.e.\ Taol) Modern Maotic no longer distinguishes between short and long vowels at all in pronunciation, and although Baerloni has no length distinction, most recent Baerloni loanwords (or ``romanizations'', so to speak) mark all open vowels with an acute accent regardless. Hence \textit{Chásur} rather than \textit{Chasur}, and names like \textit{Ólófar} and \textit{Ólúcan} rather than \textit{Olofar} and \textit{Olucan}.}

 \item \textit{Dalia}:

 \item \textit{Dálx}:

 \item \textit{Édris Elmin}: Lit. ``Elmin Mountains''.

 \item \textit{Elmincár}:

 \item \textit{Étéic}:

 \item \textit{Fírecár}:

 \item \textit{Kantane}: A Yangharad iron and copper mining town near the city of Yangharad. From Yangharad \textit{Kanta\d{n}e}.

 \item \textit{Kéí}: A mountain pass between Agrócár and Talócár, located near the Yangharad city of Uenter and the Taol city of Úbân. Modern Maotic \textipa{/kij/}, Dálx \textipa{/hij/}.

 The Yangharad name is \textit{La\.{q}ekaye}. The element \textit{la\.{q}e} translates to ``mountain pass''; \textit{Kaye} is a toponym of unclear origin, but probably has the same source as Maotic \textit{Kéí}.

 \item \textit{Loqotane}: A Yangharad iron and tin mining town near Uenter. From Yangharad \textit{\d{L}o\.{q}ota\d{n}e}.

 \item \textit{Maotel}: A city on the Maoth river. One of the largest and most important cities in Talócár, and usually recognized as the ``center'' of the Maotic empire. Classical Maotic \textipa{/m\t*{aO}tel/}, Modern \textipa{/meotel/}, Dálx \textipa{/m\t*{iu}Til/}.

 \item \textit{Maotic}: Anglicized name for the language. In Modern Maotic itself this is called \textit{Maot} (sc.\ \textit{bán}) (\textipa{/mæOt ban/}, Dálx \textipa{/m\t*{Eu}t b\~an/}). The terms \textit{Maotel bán} ``language of Maotel'' and \textit{Taol bán} ``language of the Taol'' can both be used to refer to Standard Maotic or (less commonly) to Maotic as a whole. Classical Maotic used none of these terms, but rather the single word \textit{Mást} (\textipa{/ma:st/}). The Yangharad word is \textit{(\d{l}e\d{n}om) Tiya} (\textipa{/\.*le\.*nom tij/}).

 \item \textit{Mentos}:

 \item \textit{Mittelic}:

 \item \textit{Muktai}:

 \item \textit{Myrve}:

 \item \textit{Nakhêqe}:

 \item \textit{Namó Ablés}:

 \item \textit{Qhuraal}: A mountain city in Agrócár, on the southeastern spur of the Édris Elmin, or the Tavarian language spoken there. The native pronunciation is \textipa{/q\super{h}u\*ra:l/}. Usually pronounced in Modern Maotic as \textipa{/k\super{h}0ral/}, in Dálx \textipa{/hural/}.

 \item \textit{Sevros}:

 \item \textit{Talócár}: The

 \item \textit{Taol}: Maotic name for the inhabitants of Talócár (especially Maotic speakers proper). Classical Maotic \textipa{/t\t*{aO}l/}, Modern \textipa{/tæ\s{\textltilde}/}, Dálx \textipa{/T\t*{Eu}\textltilde/}. Transparently related to the Maotic word \textit{taló} ``river''.

 \item \textit{Tavarian}:

 \item \textit{Uenter}:

 \item \textit{Wometa}: A Yangharad silver mining town near the city of Yangharad. From Yangharad \textit{Wome\d{t}a}.

 \item \textit{Yangharad}: Classical Maotic \textipa{/jaNg.harad/}; Modern \textipa{/j@N.harad/}; Dálx \textipa{/j\~\o.Nart/}. The Yangharad name is \textit{Yaŋkalat} (\textipa{/jaŋ.kawat/}), a transparent loan from Classical Maotic.

 In Classical High Tavarian the language is called \textit{Hrng} (\textipa{/h\s{\*r}ŋg/}). Tavarian dialects are, as usual, wildly divergent; reflexes include \textit{Hrn} (Qhuraal and Nakhêqe Myrve, pronunciation \textipa{/h\~{\*r}/}), \textit{Hrgh} or \textit{Hrngh} (Myrve dialects, pronunciation \textipa{/h\s{\*r}G/} or \textipa{/h\s{\~{\*r}}G/}; many Myrve dialects have already lost nasalization); \textit{H\^{r}} (Étéic Tavarian, pronunciation \textipa{/h\*r:/}), \textit{Rüng} or \textit{Ürüng} (Northwestern High Tavarian dialects, \textipa{/\*r0N/} or \textipa{/0\*r0N/}).

Classical Low Tavarian (a.k.a.\ Classical Low Mittelic) has \textit{Yaraŋ} (\textipa{/ja\*{r}aŋ/}). This survives into Ksqyidy Tavarian as \textit{Yzaŋ} (\textipa{/j\t*{za}N/}), but not into other Low Tavarian dialects, which instead borrow the Dálx term.

Classical Mittelic (a.k.a.\ Classical Inner Mittelic) has \textit{Yarŋ} (\textipa{/ja\*{r}ŋ/}). This survives into Sevrástic as \textit{Yarn} (\textipa{/jE\*{r}n/}).

 The original form seems to have been a pre-Yangharad \textit{*yarŋg} < \textit{*yargV} (meaning unknown). This was borrowed into Classical Maotic as \textit{Yang}, to which the element \textit{harad} ``plains'' was later added (for rather unclear reasons, since Yangharad is spoken in anything but plains). The pre-Yang\-harad word was also borrowed into Proto-Mittelo-Tavarian as \textit{*Yarng}, which became \textit{Hrng} by regular sound change.

 The name does not survive in the Yangharad name for the city, but does appear in some derived terms, most prominently \textit{\d{Q}e\d{l}eya\.{ŋ}} (the river running through Yangharad itself) and \textit{\d{T}e\d{n}e\d{t}eya\.{ŋ}} (a minor village outlying Yangharad).

\end{itemize}


\end{document}
